\documentclass[journal,12pt,twocolumn]{IEEEtran}
\usepackage{tikz}
\usepackage{amsmath}
\usepackage{amssymb}
\pagestyle{empty}
\usepackage{setspace}
\usepackage{gensymb}
\singlespacing

\usepackage{amsmath}
\usepackage{amsthm}
\begin{document}
\providecommand{\mbf}{\mathbf}
\providecommand{\norm}[1]{$$\left\lVert#1\right\rVert$$}

\newcommand{\myvec}[1]{\ensuremath{\begin{pmatrix}#1\end{pmatrix}}}
\let\vec\mathbf

\title{
Assignment - 1
}
\author{BILLAKURTHI SHAI SASI DEEP\\ SM21MTECH12006}
\maketitle
\newpage
\bigskip
\bibliographystyle{IEEEtran}
\section*{\textbf{Problem}}
\vspace{0.3cm}
\noindent
\textbf{Show that the following triad of points form an equilateral triangle \\
(a,0), (0,2a), (2a,a), axes being inclined at an angle of 60$^{\circ}$} 
\vspace{0.3cm}\\
\text{Answer:}  
\vspace{0.3cm}
A triangle is said to be equilateral if length of all sides are equal.
\vspace{0.6cm} 

The given points are:
\begin{align*}
\vec{A} = \myvec{a\\0}, \vec{B} =\myvec{0\\2a},
\vec{C} =\myvec{2a\\a}
\end{align*}


The distance between points A and B,
\vspace{0.3cm}
\begin{align*}
\norm{ \vec{A} - \vec{B}}&=\myvec{a\\0}-\myvec{0\\2a}=\myvec{a\\-2a}\\\\
&=\sqrt{(a)^2+(-2a)^2}\\\\
\norm{ \vec{A} - \vec{B}}&=\sqrt{5}a
\end{align*}
Similarly, 
\begin{align*}
\norm{ \vec{B} - \vec{C}}&=\myvec{0\\2a}-\myvec{2a\\a}=\myvec{-2a\\a}\\\\
&=\sqrt{(-2a)^2+(a)^2}\\\\
\norm{ \vec{B} - \vec{C}}&=\sqrt{5}a
\end{align*}
\begin{align*}
\norm{ \vec{C} - \vec{A}}&=\myvec{2a\\a}-\myvec{a\\0}=\myvec{a\\a}\\\\
&=\sqrt{(a)^2+(a)^2}\\\\
\norm{ \vec{C} - \vec{A}}&=\sqrt{2}a
\end{align*}


$\norm{ \vec{A} - \vec{B}} = \norm{ \vec{B} - \vec{C}}  \neq \norm{ \vec{C} - \vec{A}} $ \\\\
So the given triad of points does not form an equilateral triangle.

\end{document}

\end{document}
